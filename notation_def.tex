% This file determines notation
% and defines various helper commands.
% (loaded after all packages)

% some common notation
% \newcommand{\defeq}{\stackrel{\text{def}}{=}} % and my heart is broken as I
% comment this out... :(
\newcommand{\defeq}{=}

% Notation specific to Align-BDD project
\renewcommand{\vec}[1]{\mathbf{#1}} % just bold vectors, no arrows
\newcommand{\sn}[0]{\textbf{Step  \arabic{ALG@line}.}}
\newcommand{\swap}[0]{\texttt{swap}}
\newcommand{\sift}[0]{\texttt{sift}}
\newcommand{\Ninst}[0]{$10,048$}
\newcommand{\PVAL}[0]{0.6}

% some code needed for algorithms description (appendix) >>>
\newcommand{\CL}{\texttt{current-layer}}
\newcommand{\NL}{\texttt{next-layer}}
\newcommand{\IFN}{\texttt{infeasible-node}}
\newcommand{\add}[3]{\textbf{add node} to $D$: #1
  $\overset{\textrm{#2}}\longrightarrow$ \textit{(new)} #3}
\newcommand{\link}[3]{\textbf{add arc} to $D$: #1
  $\overset{\textrm{#3}}\longrightarrow$ #2 }
\newcommand{\state}{\texttt{state}}
\newcommand{\crit}{\texttt{critical-nodes}}
\newcommand{\nstate}{\texttt{next-state}}
\newcommand{\cov}{\texttt{C}}
\newcommand{\type}{\texttt{T}}
\newcommand{\hi}[1]{\texttt{HI(#1)}}
\newcommand{\lo}[1]{\texttt{LO(#1)}}

\newcommand{\F}{\textbf{F}}
\newcommand{\T}{\textbf{T}}
\newcommand{\HI}{\texttt{HI}}
\newcommand{\LO}{\texttt{LO}}
\newcommand{\ROOT}{\texttt{r}}
\newcommand{\N}{\mathcal{N}}
\newcommand{\var}{\textrm{var}}
\newcommand{\true}{\texttt{True}}
\newcommand{\false}{\texttt{False}}
\newcommand{\BN}{\mathbb{B}^N}
\newcommand{\X}{\mathbb{X}^N}
\newcommand{\orig}{\ref{probl:ap}$(A,B;T^*)$}
\newcommand{\simpl}{\ref{probl:ap}$(S_A,S_B;T_\VS^*)$}
\newcommand{\SN}{\mathbb{S}^N}
\newcommand{\VS}{\textrm{VS}}
\newcommand{\erem}{\hfill \Halmos}    % ends a remark or an example
\newcommand{\vv}{\vec{v}}

% \algdef{SE}[SUBALG]{Indent}{EndIndent}{}{\algorithmicend\ }%
% \algtext*{Indent}
% \algtext*{EndIndent}

% One style for all TikZ pictures for working with overlays:
% \tikzset{every picture/.style=remember picture}
% % Define a TikZ node for math content:
% \newcommand{\mathnode}[2]{%
%   \mathord{\tikz[baseline=(#2.base), inner sep = 0pt]{\node (#2) {$#1$};}}}

% Notation specific to DSPI algorithms notation
\newcommand{\ahat}{\widehat{\alpha}}
\newcommand{\bhat}{\widehat{\beta}}

\newcommand{\rnode}{\texttt{root}}
\newcommand{\argmin}{\textrm{argmin}}
\newcommand{\argmax}{\textrm{argmax}}
\newcommand{\und}{\textrm{ and }}
\newcommand{\isnot}[1]{\textrm{ is not }\textit{#1}} %{\not\sim\textrm{ ``#1''}}
\newcommand{\is}[1]{\sim\textrm{ ``#1''}}
\newcommand{\mrk}[1]{\texttt{mark}(#1)}

% # tree / algo specific notation
\newcommand{\actions}[1]{\texttt{actions}(#1)}
\newcommand{\children}[1]{\texttt{children}(#1)}
\newcommand{\pos}[1]{\textrm{p}(#1)}
\newcommand{\FS}[1]{\textrm{FS}(#1)}

\makeatletter
\def\LB{\@ifstar\@LB\@@LB}
\def\@LB{\textit{LB}}
\def\@@LB#1{\textit{LB}(#1)}

\def\UB{\@ifstar\@UB\@@UB}
\def\@UB{\textit{UB}}
\def\@@UB#1{\textit{UB}(#1)}
\makeatother

\newcommand{\newnode}[1]{\textrm{\textbf{create node}(#1)}}

% Horizontal algo phase section separation
\newcommand{\panelsep}{\vspace{0.7em} {\color{lightgray}\hrule} \vspace{0.7em}}
\newcommand{\psep}{\vspace{0.5em}}